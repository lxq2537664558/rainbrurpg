\section{gauge.php File Reference}
\label{gauge_8php}\index{gauge.php@{gauge.php}}


\subsection*{Functions}
\begin{CompactItemize}
\item 
{\bf draw\-Gauge} (\$caption, \$actual, \$max)
\end{CompactItemize}


\subsection{Function Documentation}
\index{gauge.php@{gauge.php}!drawGauge@{drawGauge}}
\index{drawGauge@{drawGauge}!gauge.php@{gauge.php}}
\subsubsection{\setlength{\rightskip}{0pt plus 5cm}draw\-Gauge (\$ {\em caption}, \$ {\em actual}, \$ {\em max})}\label{gauge_8php_3c04d1c808721c741013f290fdf6a501}


This functions create a gauge with 2 values\begin{itemize}
\item \$actual : The value to indicate\item \$max : The max number\item \$caption: The caption of the gauge\end{itemize}


If \$actual=\$max, the gauge is at 100\%.

It is highly customizable :\begin{itemize}
\item \$gauge\-Png The filename of the image of a gauge pixel width;\item \$back\-Png The filename of the image of the background of the gauge.\end{itemize}


\begin{itemize}
\item \$qaugewidth : The width of the gauge in pixels. \end{itemize}
